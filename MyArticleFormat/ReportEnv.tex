

% ----------------------------------------------常用packege
\usepackage{latexsym}
\usepackage{amsmath,amssymb}
\usepackage{mathtools}
\usepackage[x11names]{xcolor}
\usepackage{graphicx}
\usepackage{algorithm}
\usepackage{amsthm}
\usepackage{lmodern} % 解決 font warning
\usepackage{animate} % insert gif

\usepackage{lipsum} % To generate test text 
\usepackage{ulem} % 下劃線,波浪線

\usepackage{listings} % display code on slides; don't forget [fragile] option after \begin{frame}
% ----------------------------------------------
\usepackage[left=2.5cm, right=2.5cm, top=1.5cm, bottom=2.5cm]{geometry}
\usepackage{float}
% \usepackage{subfigure}
\usepackage{caption}
\usepackage{array}
\usepackage{bm}
\newcommand{\bmrm}[1]{\bm{\mathrm{#1}}}
\usepackage{algorithmicx}
\usepackage{algpseudocode}
\usepackage{hyperref}
\usepackage{animate}
\usepackage{movie15}
\usepackage{textcomp}
\usepackage{cite}
\usepackage{subfig}
\usepackage{makecell}
\usepackage{kantlipsum}% dummy text
\usepackage{stfloats}

% 縮排
\usepackage{indentfirst} 
\setlength{\parindent}{2em}

% 行距
\linespread{1.4}\selectfont

% xeCJK
\usepackage{xeCJK}

% 2021-11-08 自訂環境
\newenvironment{ee} % 自訂矩陣環境
	{\begin{equation}\begin{aligned}}{\end{aligned}\end{equation}}

\newenvironment{bmat} % 自訂矩陣環境
	{\begin{bmatrix}}{\end{bmatrix}}

% figure 設定
% ------ figure 路徑 ------
\graphicspath{{figure}} 
% ------ figure tag ------
\newcommand{\figuretag}[1]{%
  \addtocounter{figure}{-1}%
  \renewcommand{\thefigure}{#1}%
}
\captionsetup[figure]{labelfont={bf},name={Fig.},labelsep=period}
% ------ table tag ------
% \newcommand{\tabletag}[1]{%
%     \addtocounter{table}{-1}%
%     \renewcommand{\thetable}{#1}%
% }
% \captionsetup[table]{labelfont={bf},name={\kaishu 表},labelsep=period}

% 顯示程式碼設定
\usepackage{listings} % display code on slides; don't forget [fragile] option after \begin{frame}    
  \lstset{inputpath={code}}
  
  \definecolor{Blue}{rgb}{0,0,1}
  \definecolor{Green}{rgb}{0,0.5,0}
  \definecolor{Red}{rgb}{0.64,0.08,0.08}

  \lstdefinestyle{C++}{
      language=C++,
      captionpos=b,                       % 讓Caption在Bottom的位置
      numbers=left,                       % 程式碼行號
      frame=lines,                        %開始結束有條黑線
      showstringspaces=false,             % "不"標註空格
      escapeinside={(@}{@)},            % 脫逃字元
      commentstyle=\color{Green},         % Comment顏色
      keywordstyle=\color{Blue},          % Keyword顏色
      stringstyle=\color{Red},            % String顏色
      basicstyle=\ttfamily\small,         % 字型
  }

  \definecolor{mygreen}{RGB}{28,172,0} % color values Red, Green, Blue
  \definecolor{mylilas}{RGB}{170,55,241}

  \lstdefinestyle{Matlab}{language=Matlab,%
  %basicstyle=\color{red},
  basicstyle=\ttfamily\small,         % 字型
  frame=lines,%開始結束有條黑線
  morekeywords={matlab2tikz},
  keywordstyle=\color{blue},%
  morekeywords=[2]{1}, keywordstyle=[2]{\color{black}},
  identifierstyle=\color{black},%
  stringstyle=\color{mylilas},
  commentstyle=\color{mygreen},%
  showstringspaces=false,%without this there will be a symbol in the places where there is a space
  numbers=left,%
  numberstyle={\tiny \color{black}},% size of the numbers
  numbersep=9pt, % this defines how far the numbers are from the text
  emph=[1]{for,end,break},emphstyle=[1]\color{red}, %some words to emphasise
  %emph=[2]{word1,word2}, emphstyle=[2]{style},    
  }

%% matlab code block example 
% \begin{lstinputlisting}[style=Matlab]{main.m}
% \end{lstinputlisting}

%% C++ code block example
% \begin{lstlisting}[style=C++]
%   #include <iostream>
%   int main()
%   {
%       // Print Hello (@Test escapeinside@)
%       std::cout << "Hello, world!" << std::endl;
%       return 0;
%   }
% \end{lstlisting}


