\documentclass[10pt,xcolor={x11names}]{beamer}
% Chih, Chao-Hsien
% 2022-02-21 Ver.1


%------------------------------------------------------------


% \setbeamercolor*{palette tertiary}{fg=black,bg=black!10} % 最上面(Section)的眉標
% \setbeamercolor*{palette quaternary}{fg=black,bg=black!10} 

% footline (\insert 自定義)
% \defbeamertemplate*{footline}{mytheme}
% {
%   \leavevmode%
%   \hbox{%
%   \begin{beamercolorbox}[wd=.3\paperwidth,ht=2.25ex,dp=1ex,center]{author in head/foot}%
%     \usebeamerfont{author in head/foot}~\insertshortinstitute{}
%   \end{beamercolorbox}%
%   \begin{beamercolorbox}[wd=.7\paperwidth,ht=2.25ex,dp=1ex,right]{date in head/foot}%
%     \usebeamerfont{date in head/foot}\insertshortauthor{}\hspace*{20em}
%     \insertframenumber{} / \inserttotalframenumber\hspace*{2ex} 
%   \end{beamercolorbox}}%
%   \vskip0pt%
% }
% \usebeamertemplate{mytheme}

% ----------------------------------------------常用packege
\usepackage{latexsym}
\usepackage{amsmath,amssymb}
\usepackage{mathtools}

\usepackage{algorithm}
\usepackage{amsthm}
\usepackage{lmodern} % 解决 font warning

\usepackage{animate} % insert gif

\usepackage{lipsum} % To generate test text 
\usepackage{ulem} % 下划线,波浪线

\usepackage{listings} % display code on slides; don't forget [fragile] option after \begin{frame}


% 2021-11-08
\usepackage{graphicx} % 插入圖片 
\graphicspath{{Figure/}} % 指定圖檔搜尋路徑
\usepackage{fontspec} % 加這個就可以設定字體。
\usepackage{xeCJK} % 讓中英文字體分開設置。
%設定中文為系統上的字型,而英文不去更動,使用原TEX字型。
\setCJKmainfont[AutoFakeBold=2]{MSJH.TTC}
\setCJKsansfont[AutoFakeBold=2]{MSJH.TTC}
%\setCJKmainfont[BoldFont={王漢宗特黑體繁}]{王漢宗細黑體繁}
%\XeTeXlinebreaklocale "zh" %這兩行一定要加,中文才能自動換行。
%\XeTeXlinebreakskip = 0pt plus 1pt % 這兩行一定要加,中文才能自動換行。

\usepackage{float}
\usepackage{subfigure}
\usepackage{array}
\usepackage{bm}
\newcommand{\bmrm}[1]{\bm{\mathrm{#1}}}
\usepackage{xcolor}
\usepackage{bookmark}
\usepackage{algorithmicx}
\usepackage{algpseudocode}
\usepackage{hyperref}
\usepackage{animate}
\usepackage{movie15}


% 設置英文字體
% \setsansfont[
%     BoldFont=TIMESBD.TTF, 
%     ItalicFont=TIMESI.TTF, 
%     BoldItalicFont=TIMESBI.TTF 
%     ]{TIMES.TTF}

% ----------------------------------------------

% Define the environment which puts the frame
% In this case, the environment also accepts an argument with an optional
% title (which defaults to ``Example'', which is typeset in a box overlaid
% on the top border
\newenvironment{parchment}[1][Example]{%
  \def\FrameCommand{\parchmentframe}%
  \def\FirstFrameCommand{\parchmentframetop}%
  \def\LastFrameCommand{\parchmentframebottom}%
  \def\MidFrameCommand{\parchmentframemiddle}%
  \vskip\baselineskip
  \MakeFramed {\FrameRestore}
  \noindent\tikz\node[inner sep=1ex, draw=black!20, fill=black!90, 
          anchor=west, overlay] at (0em, 2em) {\sffamily#1};\par}%
{\endMakeFramed}

% ---------------------------------------------------------------------
% ---------------------------------------------------------------------

% code setting
\lstset{
    language=C++,
    basicstyle=\ttfamily\footnotesize,
    keywordstyle=\color{red},
    breaklines=true,
    xleftmargin=2em,
    numbers=left,
    numberstyle=\color[RGB]{222,155,81},
    frame=leftline,
    tabsize=4,
    breakatwhitespace=false,
    showspaces=false,               
    showstringspaces=false,
    showtabs=false,
    morekeywords={Str, Num, List},
}

% ---------------------------------------------------------------------

\newcommand{\reditem}[1]{\setbeamercolor{item}{fg=red}\item #1}

% 缩放公式大小
\newcommand*{\Scale}[2][4]{\scalebox{#1}{\ensuremath{#2}}}

% 解决 font warning
\renewcommand\textbullet{\ensuremath{\bullet}}

% 圖片路徑
\graphicspath{{figure}}

% -------------------------------------------------------------

% --------------------------------------------------------------
% theme
\usetheme{Productive}

\setlength{\parskip}{0.5em}


\DeclareMathOperator{\sign}{sign} % 自訂rank正體語法
\DeclareMathOperator{\opt}{opt} % 自訂rank正體語法

\newcommand{\ui}{u_{1}}
\newcommand{\suii}{s_{u_{2}}}
\newcommand{\cuii}{c_{u_{2}}}
\newcommand{\suiii}{s_{u_{3}}}
\newcommand{\cuiii}{c_{u_{3}}}
\newcommand{\sphi}{s_{\phi}}
\newcommand{\cphi}{c_{\phi}}
\newcommand{\stheta}{s_{\theta}}
\newcommand{\ctheta}{c_{\theta}}
\newcommand{\spsi}{s_{\psi}}
\newcommand{\cpsi}{c_{\psi}}

\AtBeginSection[]
{
    \begin{frame}<beamer>
        \frametitle{Outline}
        \tableofcontents[currentsection]
    \end{frame}
}

%-----------------------------------------------------------------------------
% DOCUMENT PROPERTIES

\title[Short Title]{Title}

\author[P46104269@gs.ncku.edu.tw]{\textbf{Student:} Chao-Hsien Chih \and \textbf{Professor:} Chao-Chung Peng }

\institute[NCKU-IAA IEC-Lab] % (optional, but mostly needed)
{

    \begin{tabular}{cc}
        \begin{minipage}{0.15\linewidth}
            \begin{figure}
                \includegraphics[width = \linewidth]{IEC_Lab_Logo.png}
            \end{figure}
        \end{minipage}
        &
        \begin{minipage}{0.6\linewidth}
            Intelligent Embedded Control Lab (IEC-Lab) \\
            Department of Aeronautics and Astronautics \\
            National Cheng Kung University \\
            Tainan, Taiwan 
        \end{minipage}
    \end{tabular}
}
\date{\today}

% -------------------------------------------------------------


\begin{document}

\begin{frame}
    \maketitle
\end{frame}

\begin{frame}{Outline}
    \tableofcontents
\end{frame}

\section{Review of Prior Studies and Literature}

\begin{frame}{Review of Prior Studies and Literature}
    \begin{block}{The Coaxial Rotor UAV}
        \begin{itemize}
            \item 123
        \end{itemize}
    \end{block}
    \begin{alertblock}{The Swashplateless MAV}
        \begin{itemize}
            \item 123
        \end{itemize}
    \end{alertblock}
    \begin{exampleblock}{The Vectored-Thrust Coaxial UAV}
        \begin{itemize}
            \item 123
        \end{itemize}
    \end{exampleblock}
\end{frame}

\begin{frame}{Review of Prior Studies and Literature}
    \begin{columns}
        \begin{column}{0.5\linewidth}
            \begin{block}{The Coaxial Rotor UAV}
                \begin{figure}
                    \centering
                    \includegraphics[width=0.7\linewidth]{Prior_1.png}
                \end{figure}
            \end{block}
            \begin{alertblock}{The Swashplateless MAV}
                \begin{figure}
                    \centering
                    \includegraphics[width=0.7\linewidth]{Prior_2.png}
                \end{figure}
            \end{alertblock}
        \end{column}
        \begin{column}{0.5\linewidth}
            \begin{exampleblock}{The Vectored-Thrust Coaxial UAV}
                \begin{figure}
                    \centering
                    \includegraphics[width=0.7\linewidth]{Prior_3.png}
                \end{figure}
            \end{exampleblock}
        \end{column}
    \end{columns}
\end{frame}

\begin{frame}{LM-based Optimal Algorithm}
    \begin{center}
        \begin{minipage}{\linewidth}
            \begin{algorithm}[H]
                \caption{LM-based Optimal Control Allocation Algorithm}\label{alg:LM}
                \begin{algorithmic}
                    \State \textbf{given} an initial value $\bmrm{u}^{(0)}$, $\lambda^{(0)} = 1000$, $\epsilon = 10^{-5}$.
                    \Repeat
                    \State 1.~Determine a Jacobian matrix $\bmrm{J}_{\bmrm{r}}^{(k)}$.
                    \State 2. Update the damping parameter $\lambda^{(k)}$.
                    \State 3.~Update the LM step. 
                    \State ~~~~$\bmrm{d}^{(k)} = -\left({\bmrm{J}_{\bmrm{r}}^{(k)T}} \bmrm{J}_{\bmrm{r}}^{(k)} +\lambda^{(k)} {\bmrm{I}_4} \right)^{-1} \bmrm{J}_{\bmrm{r}}^{(k)T} \bmrm{r}(\bmrm{u}^{(k)})$.
                    \State 4.~Update the control variables.
                    \State ~~~~$\bmrm{u}^{(k+1)} = \bmrm{u}^{(k)} +\bmrm{d}^{(k)}$.
                    \State ~~~~$k \leftarrow k+1$.
                    \Until{$\| \bmrm{r} \| < \epsilon $ is satisfied, $\bmrm{u}^{*}=\bmrm{u}^{(k+1)}$.}
                \end{algorithmic}
            \end{algorithm}
        \end{minipage}
    \end{center}
    \begin{itemize}
        \item In the first iteration step, the initial value $\bmrm{u}^{(0)}$ will be set to zero. After, the initial value is set to the solved result of the previous step.
    \end{itemize}
\end{frame}

\begin{frame}{LM-based Optimal Algorithm}
    \begin{itemize}
        \item The condition number $\mathcal{C}$ of the matrix $\left({\bmrm{J}_{\bmrm{r}}^{(k)T}} \bmrm{J}_{\bmrm{r}}^{(k)} +\lambda \bmrm{I}_3\right)$ is calculated, and adjust $\lambda$ adaptively:
        \begin{equation}
            \lambda=
            \begin{cases}
                1000,  &\mathcal{C} \geq 10^5\\
                0.001, &\mathcal{C} < 10^5
            \end{cases}
        \end{equation}
        \item When the reduction of {\color{Red1}the cost function is rapid}, a {\color{Red1}smaller value} can be applied to accelerate the speed of converting. On the other hand, if the matrix $\left({\bmrm{J}_{\bmrm{r}}^{(k)T}} \bmrm{J}_{\bmrm{r}}^{(k)} +\lambda \bmrm{I}_3\right)$ is {\color{Red1}ill-condition} to introduce the numerical errors, the {\color{Red1}larger value} is used to converge with the small gradient step.
    \end{itemize}
\end{frame}


\end{document}