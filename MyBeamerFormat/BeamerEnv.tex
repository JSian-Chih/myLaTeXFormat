% Chih, Chao-Hsien
% 2022-02-21 Ver.1


%------------------------------------------------------------


% \setbeamercolor*{palette tertiary}{fg=black,bg=black!10} % 最上面(Section)的眉標
% \setbeamercolor*{palette quaternary}{fg=black,bg=black!10} 

% footline (\insert 自定義)
% \defbeamertemplate*{footline}{mytheme}
% {
%   \leavevmode%
%   \hbox{%
%   \begin{beamercolorbox}[wd=.3\paperwidth,ht=2.25ex,dp=1ex,center]{author in head/foot}%
%     \usebeamerfont{author in head/foot}~\insertshortinstitute{}
%   \end{beamercolorbox}%
%   \begin{beamercolorbox}[wd=.7\paperwidth,ht=2.25ex,dp=1ex,right]{date in head/foot}%
%     \usebeamerfont{date in head/foot}\insertshortauthor{}\hspace*{20em}
%     \insertframenumber{} / \inserttotalframenumber\hspace*{2ex} 
%   \end{beamercolorbox}}%
%   \vskip0pt%
% }
% \usebeamertemplate{mytheme}

% ----------------------------------------------常用packege
\usepackage{latexsym}
\usepackage{amsmath,amssymb}
\usepackage{mathtools}

\usepackage{algorithm}
\usepackage{amsthm}
\usepackage{lmodern} % 解决 font warning

\usepackage{animate} % insert gif

\usepackage{lipsum} % To generate test text 
\usepackage{ulem} % 下划线,波浪线

\usepackage{listings} % display code on slides; don't forget [fragile] option after \begin{frame}


% 2021-11-08
\usepackage{graphicx} % 插入圖片 
\graphicspath{{Figure/}} % 指定圖檔搜尋路徑
\usepackage{fontspec} % 加這個就可以設定字體。
\usepackage{xeCJK} % 讓中英文字體分開設置。
%設定中文為系統上的字型,而英文不去更動,使用原TEX字型。
\setCJKmainfont[AutoFakeBold=2]{MSJH.TTC}
\setCJKsansfont[AutoFakeBold=2]{MSJH.TTC}
%\setCJKmainfont[BoldFont={王漢宗特黑體繁}]{王漢宗細黑體繁}
%\XeTeXlinebreaklocale "zh" %這兩行一定要加,中文才能自動換行。
%\XeTeXlinebreakskip = 0pt plus 1pt % 這兩行一定要加,中文才能自動換行。

\usepackage{float}
% \usepackage{subfigure}
\usepackage{array}
\usepackage[caption=false,font=normalsize,labelfont=rm,textfont=rm]{subfig}
\usepackage{bm}
\newcommand{\bmrm}[1]{\bm{\mathrm{#1}}}
\usepackage{bookmark}
\usepackage{algorithmicx}
\usepackage{algpseudocode}
\usepackage{hyperref}
\usepackage{animate}
\usepackage{movie15}
\usepackage{url}
\usepackage{cite}

\usepackage{tcolorbox}

\setbeamertemplate{bibliography item}[text]
% 設置英文字體
% \setsansfont[
%     BoldFont=TIMESBD.TTF, 
%     ItalicFont=TIMESI.TTF, 
%     BoldItalicFont=TIMESBI.TTF 
%     ]{TIMES.TTF}

% ----------------------------------------------

% Define the environment which puts the frame
% In this case, the environment also accepts an argument with an optional
% title (which defaults to ``Example'', which is typeset in a box overlaid
% on the top border
\newenvironment{parchment}[1][Example]{%
  \def\FrameCommand{\parchmentframe}%
  \def\FirstFrameCommand{\parchmentframetop}%
  \def\LastFrameCommand{\parchmentframebottom}%
  \def\MidFrameCommand{\parchmentframemiddle}%
  \vskip\baselineskip
  \MakeFramed {\FrameRestore}
  \noindent\tikz\node[inner sep=1ex, draw=black!20, fill=black!90, 
          anchor=west, overlay] at (0em, 2em) {\sffamily#1};\par}%
{\endMakeFramed}

% ---------------------------------------------------------------------
% ---------------------------------------------------------------------

% code setting
\lstset{
    language=C++,
    basicstyle=\ttfamily\footnotesize,
    keywordstyle=\color{red},
    breaklines=true,
    xleftmargin=2em,
    numbers=left,
    numberstyle=\color[RGB]{222,155,81},
    frame=leftline,
    tabsize=4,
    breakatwhitespace=false,
    showspaces=false,               
    showstringspaces=false,
    showtabs=false,
    morekeywords={Str, Num, List},
}

% ---------------------------------------------------------------------

\newcommand{\reditem}[1]{\setbeamercolor{item}{fg=red}\item #1}

% 缩放公式大小
\newcommand*{\Scale}[2][4]{\scalebox{#1}{\ensuremath{#2}}}

% 解决 font warning
\renewcommand\textbullet{\ensuremath{\bullet}}

% 圖片路徑
\graphicspath{{figure}}

% -------------------------------------------------------------

% --------------------------------------------------------------
% theme
\usetheme{Productive}

\setlength{\parskip}{0.5em}

